\documentclass[a4paper]{article}
\usepackage[T1]{fontenc}
\usepackage[utf8]{inputenc}
\usepackage[left=2.5cm, right=2.5cm, top=2.5cm, bottom=2.5cm]{geometry}
\usepackage[breaklinks, hidelinks]{hyperref}
\usepackage{xcolor}
\usepackage{titlesec}
\usepackage{tocloft}
\usepackage{fancyhdr}
\usepackage{ifthen}

\titleformat*{\section}{\large\bfseries}
\titleformat*{\subsection}{\normalsize\bfseries}

\setcounter{tocdepth}{1}
\renewcommand{\cftsecfont}{\normalfont}
\renewcommand{\cftsecpagefont}{\normalfont}
\renewcommand{\cftsecleader}{\cftdotfill{\cftdotsep}}
\setlength{\cftsecindent}{0.5cm}
\let\oldpart\part
\newcommand{\parttitle}{}
\renewcommand{\part}[1]{\oldpart{#1}\renewcommand{\parttitle}{#1}}

\pagestyle{fancy}
\lhead{\ifthenelse{\equal{\thepart}{}}{Contents}{Part \thepart~---~\parttitle}}
\rhead{Joachim Favre \& Alberts Reisons}

\title{A set of proofs that definitely deserve a 6}
\author{Joachim Favre \& Alberts Reisons}
\date{6th May 2021}

\begin{document}
\maketitle
\tableofcontents
\newpage

\part{Axioms}


\section{The removal of parenthesis\label{6}}
We assume this equality:
\[\left(a\right) = a\]
with $a$ being unknown or known.


\section{The power distribution\label{7}}
We suppose the following equality:
\[b^x  b^y = b^{\left(x + y\right)}\]
with $b$, $x$ and $y$ being unknown (or known).


\section{The identity exponent\label{8}}
We suppose this proposition:
\[a^1 = a\]
with $a$ being unknown (or known).


\section{The commutativity of the product\label{10}}
We assume this proposition to be true:
\[ab = ba\]
with $a$ and $b$ being unknown (or known).


\section{The left distributivity of the product\label{11}}
We take the following affirmation as an axiom:
\[a\left(b + c\right) = ab + ac\]
with $a$, $b$ and $c$ being unknown (or known).


\section{The left multiplication by the identity\label{12}}
We suppose that this proposition is true:
\[1a = a\]
with $a$ being unknown (or known).


\section{The commutativity of the addition\label{20}}
We take the following proposition as an axiom:
\[a + b = b + a\]
with $a$ and $b$ being unknown (or known).
\newpage

\part{Theorems}

\section{The addition\label{0}}
\subsection{Theorem}
We are trying to prove this affirmation:
\[a + b = c\]
with $a$ and $b$ being unknown (or known), and with $c=a+b$ getting simplified (as a number).\subsection{Proof}
Let $a$ and $b$ be unknown (or known). Let $c=a+b$ be a simplification (as a number). Starting with the following expression:
\[a+b\]

We have let $a+b=c$, so
\[a+b=c\]

This proof thus allows us to be convinced that:
\[a+b = c\]
\begin{flushright}
QED
\end{flushright}



\section{The product\label{1}}
\subsection{Theorem}
We are aiming to prove the following equality:
\[a  b = c\]
with $a$ and $b$ being unknown (or known), and with $c=ab$ getting simplified (as a number).\subsection{Proof}
Let $a$ and $b$ be unknown (or known). Let $c=ab$ be a simplification (as a number). Let us begin with:
\[ab\]

We have let $ab=c$, so
\[ab=c\]

Which thus makes us sure that:
\[ab = c\]
\begin{flushright}
QED
\end{flushright}



\section{The power\label{2}}
\subsection{Theorem}
We are aiming to show that:
\[a^b = c\]
with $a$ and $b$ being unknown (or known), and with $c=a^b$ getting simplified (as a number).\subsection{Proof}
Let $a$ and $b$ be unknown (or known). Let $c=a^b$ be a simplification (as a number). Starting with the following expression:
\[a^b\]

We have let $a^b=c$, so
\[a^b=c\]

Which thus makes us sure:
\[a^b = c\]
\begin{flushright}
QED
\end{flushright}



\section{The distribution of the square\label{5}}
\subsection{Theorem}
We aim to show this proposition:
\[a^2 = aa\]
with $a$ being unknown (or known).\subsection{Proof}
Let $a$ be unknown (or known). The following expression can be our starting point:
\[a^2\]

According to the removal of parenthesis (section \ref{6}), we know that $2=\left(2\right)$. Therefore,
\[a^2=a^{\left(2\right)}\]

From the addition (section \ref{0}), we get that $2=1+1$. This means that:
\[a^{\left(2\right)}=a^{\left(1+1\right)}\]

From the power distribution (section \ref{7}), we have that
\[a^{\left(1+1\right)}=a^1a^1\]

By the identity exponent (section \ref{8}), we have that $a^1=a$. Therefore,
\[a^1a^1=aa^1\]

From the identity exponent (section \ref{8}), we know that $a^1=a$. This means that:
\[aa^1=aa\]

This proof thus makes us sure:
\[a^2 = aa\]
\begin{flushright}
QED
\end{flushright}



\section{The right distributivity of the product\label{9}}
\subsection{Theorem}
We aim to prove this equality:
\[\left(a + b\right)c = ac + bc\]
with $a$, $b$ and $c$ being unknown (or known).\subsection{Proof}
Let $a$, $b$ and $c$ be unknown (or known). Let us start with:
\[\left(a+b\right)c\]

Using the commutativity of the product (section \ref{10}), we have
\[\left(a+b\right)c=c\left(a+b\right)\]

Using the left distributivity of the product (section \ref{11}), we get
\[c\left(a+b\right)=ca+cb\]

From the commutativity of the product (section \ref{10}), we know that $ca=ac$. This allows us to infer that:
\[ca+cb=ac+cb\]

Using the commutativity of the product (section \ref{10}), we have $cb=bc$. Thus,
\[ac+cb=ac+bc\]

This proof makes us sure:
\[\left(a+b\right)c = ac+bc\]
\begin{flushright}
QED
\end{flushright}



\section{The litteral addition\label{13}}
\subsection{Theorem}
We are trying to prove the following affirmation:
\[ax + bx = cx\]
with $a$, $b$ and $x$ being unknown (or known), and with $c=a+b$ getting simplified (as a number).\subsection{Proof}
Let $a$, $b$ and $x$ be unknown (or known). Let $c=a+b$ be a simplification (as a number). We will begin our proof with the following expression:
\[ax+bx\]

By the right distributivity of the product (section \ref{9}), we have that
\[ax+bx=\left(a+b\right)x\]

We have let $a+b=c$, so
\[\left(a+b\right)x=\left(c\right)x\]

From the removal of parenthesis (section \ref{6}), we get $\left(c\right)=c$. Therefore,
\[\left(c\right)x=cx\]

This proof therefore allows us to conclude that:
\[ax+bx = cx\]
\begin{flushright}
QED
\end{flushright}



\section{The first remarkable identity\label{4}}
\subsection{Theorem}
We are trying to prove the following equality:
\[\left(a+b\right)^2 = a^2 + 2ab + b^2\]
with $a$ and $b$ being unknown (or known).\subsection{Proof}
Let $a$ and $b$ be unknown (or known). Let us take this as a starting point:
\[\left(a+b\right)^2\]

Using the distribution of the square (section \ref{5}), we have
\[\left(a+b\right)^2=\left(a+b\right)\left(a+b\right)\]

According to the right distributivity of the product (section \ref{9}), we get that
\[\left(a+b\right)\left(a+b\right)=a\left(a+b\right)+b\left(a+b\right)\]

Using the left distributivity of the product (section \ref{11}), we get that $a\left(a+b\right)=aa+ab$. This allows us to infer that:
\[a\left(a+b\right)+b\left(a+b\right)=aa+ab+b\left(a+b\right)\]

By the left distributivity of the product (section \ref{11}), we have $b\left(a+b\right)=ba+bb$. This allows us to infer that:
\[aa+ab+b\left(a+b\right)=aa+ab+ba+bb\]

From the commutativity of the product (section \ref{10}), we get $ba=ab$. Thus,
\[aa+ab+ba+bb=aa+ab+ab+bb\]

By the left multiplication by the identity (section \ref{12}), we have $ab=1ab$. This means that:
\[aa+ab+ab+bb=aa+1ab+ab+bb\]

From the left multiplication by the identity (section \ref{12}), we know that $ab=1ab$. Thus,
\[aa+1ab+ab+bb=aa+1ab+1ab+bb\]

From the litteral addition (section \ref{13}), we get $1ab+1ab=2ab$. Thus,
\[aa+1ab+1ab+bb=aa+2ab+bb\]

Using the distribution of the square (section \ref{5}), we get that $aa=a^2$. This means that:
\[aa+2ab+bb=a^2+2ab+bb\]

By the distribution of the square (section \ref{5}), we know that $bb=b^2$. This means that:
\[a^2+2ab+bb=a^2+2ab+b^2\]

This proof therefore makes us sure:
\[\left(a+b\right)^2 = a^2+2ab+b^2\]
\begin{flushright}
QED
\end{flushright}



\section{The right multiplication by the identity\label{14}}
\subsection{Theorem}
We aim to prove this affirmation:
\[a\cdot1 = a\]
with $a$ being unknown (or known).\subsection{Proof}
Let $a$ be unknown (or known). Let us take this as a starting point:
\[a\cdot1\]

Using the commutativity of the product (section \ref{10}), we know that
\[a\cdot1=1a\]

Using the left multiplication by the identity (section \ref{12}), we know that
\[1a=a\]

Which allows us to conclude:
\[a\cdot1 = a\]
\begin{flushright}
QED
\end{flushright}



\section{The first identity with a twist\label{3}}
\subsection{Theorem}
We are trying to prove the following equality:
\[\left(x + 1\right)^2 = x^2 + 2x + 1\]
with $x$ being unknown (or known).\subsection{Proof}
Let $x$ be unknown (or known). We can begin with:
\[\left(x+1\right)^2\]

According to the first remarkable identity (section \ref{4}), we get
\[\left(x+1\right)^2=x^2+2x\cdot1+1^2\]

Using the right multiplication by the identity (section \ref{14}), we have that $2x\cdot1=2x$. Thus,
\[x^2+2x\cdot1+1^2=x^2+2x+1^2\]

From the power (section \ref{2}), we have $1^2=1$. This means that:
\[x^2+2x+1^2=x^2+2x+1\]

This proof allows us to conclude:
\[\left(x+1\right)^2 = x^2+2x+1\]
\begin{flushright}
QED
\end{flushright}



\section{The triple left distributivity of the product\label{16}}
\subsection{Theorem}
We are aiming to prove this affirmation:
\[a\left(b + c + d\right) = ab + ac + ad\]
with $a$, $b$, $c$ and $d$ being unknown (or known).\subsection{Proof}
Let $a$, $b$, $c$ and $d$ be unknown (or known). Let us take this as a starting point:
\[a\left(b+c+d\right)\]

Using the removal of parenthesis (section \ref{6}), we know that $c+d=\left(c+d\right)$. This allows us to infer that:
\[a\left(b+c+d\right)=a\left(b+\left(c+d\right)\right)\]

From the left distributivity of the product (section \ref{11}), we have
\[a\left(b+\left(c+d\right)\right)=ab+a\left(c+d\right)\]

From the left distributivity of the product (section \ref{11}), we get that $a\left(c+d\right)=ac+ad$. Thus,
\[ab+a\left(c+d\right)=ab+ac+ad\]

Which thus allows us to conclude:
\[a\left(b+c+d\right) = ab+ac+ad\]
\begin{flushright}
QED
\end{flushright}



\section{The triple right distributivity of the product\label{15}}
\subsection{Theorem}
We want to show this equality:
\[\left(a + b + c\right)d = ad + bd + cd\]
with $a$, $b$, $c$ and $d$ being unknown (or known).\subsection{Proof}
Let $a$, $b$, $c$ and $d$ be unknown (or known). Let us take this as a starting point:
\[\left(a+b+c\right)d\]

According to the commutativity of the product (section \ref{10}), we get
\[\left(a+b+c\right)d=d\left(a+b+c\right)\]

According to the triple left distributivity of the product (section \ref{16}), we get
\[d\left(a+b+c\right)=da+db+dc\]

Using the commutativity of the product (section \ref{10}), we get $da=ad$. This means that:
\[da+db+dc=ad+db+dc\]

By the commutativity of the product (section \ref{10}), we get $db=bd$. Therefore,
\[ad+db+dc=ad+bd+dc\]

By the commutativity of the product (section \ref{10}), we get that $dc=cd$. Therefore,
\[ad+bd+dc=ad+bd+cd\]

This proof allows us to conclude that:
\[\left(a+b+c\right)d = ad+bd+cd\]
\begin{flushright}
QED
\end{flushright}



\section{The left distribution of the cube\label{18}}
\subsection{Theorem}
We want to show this proposition:
\[a^3 = a^2a\]
with $a$ being unknown (or known).\subsection{Proof}
Let $a$ be unknown (or known). Starting with the following expression:
\[a^3\]

According to the removal of parenthesis (section \ref{6}), we get $3=\left(3\right)$. This means that:
\[a^3=a^{\left(3\right)}\]

According to the addition (section \ref{0}), we have that $3=2+1$. This allows us to infer that:
\[a^{\left(3\right)}=a^{\left(2+1\right)}\]

From the power distribution (section \ref{7}), we know that
\[a^{\left(2+1\right)}=a^2a^1\]

From the identity exponent (section \ref{8}), we have that $a^1=a$. Therefore,
\[a^2a^1=a^2a\]

Which makes us sure:
\[a^3 = a^2a\]
\begin{flushright}
QED
\end{flushright}



\section{The right distribution of the cube\label{17}}
\subsection{Theorem}
We are trying to prove the following proposition:
\[a^3 = aa^2\]
with $a$ being unknown (or known).\subsection{Proof}
Let $a$ be unknown (or known). A friend of mine told me to start with:
\[a^3\]

According to the left distribution of the cube (section \ref{18}), we get that
\[a^3=a^2a\]

According to the commutativity of the product (section \ref{10}), we have
\[a^2a=aa^2\]

Which therefore makes us sure:
\[a^3 = aa^2\]
\begin{flushright}
QED
\end{flushright}



\section{The cube remarkable identity\label{19}}
\subsection{Theorem}
We are trying to prove the following affirmation:
\[\left(a + b\right)^3 = a^3 + 3a^2b + 3ab^2 + b^3\]
with $a$ and $b$ being unknown (or known).\subsection{Proof}
Let $a$ and $b$ be unknown (or known). We can begin with:
\[\left(a+b\right)^3\]

According to the left distribution of the cube (section \ref{18}), we get that
\[\left(a+b\right)^3=\left(a+b\right)^2\left(a+b\right)\]

By the removal of parenthesis (section \ref{6}), we get $\left(a+b\right)^2=\left(\left(a+b\right)^2\right)$. This allows us to infer that:
\[\left(a+b\right)^2\left(a+b\right)=\left(\left(a+b\right)^2\right)\left(a+b\right)\]

Using the first remarkable identity (section \ref{4}), we know that $\left(a+b\right)^2=a^2+2ab+b^2$. This means that:
\[\left(\left(a+b\right)^2\right)\left(a+b\right)=\left(a^2+2ab+b^2\right)\left(a+b\right)\]

From the left distributivity of the product (section \ref{11}), we have
\[\left(a^2+2ab+b^2\right)\left(a+b\right)=\left(a^2+2ab+b^2\right)a+\left(a^2+2ab+b^2\right)b\]

From the triple right distributivity of the product (section \ref{15}), we get $\left(a^2+2ab+b^2\right)a=a^2a+2aba+b^2a$. Therefore,
\[\left(a^2+2ab+b^2\right)a+\left(a^2+2ab+b^2\right)b=a^2a+2aba+b^2a+\left(a^2+2ab+b^2\right)b\]

From the triple right distributivity of the product (section \ref{15}), we get that $\left(a^2+2ab+b^2\right)b=a^2b+2abb+b^2b$. This allows us to infer that:
\[a^2a+2aba+b^2a+\left(a^2+2ab+b^2\right)b=a^2a+2aba+b^2a+a^2b+2abb+b^2b\]

Using the left distribution of the cube (section \ref{18}), we get $a^2a=a^3$. Therefore,
\[a^2a+2aba+b^2a+a^2b+2abb+b^2b=a^3+2aba+b^2a+a^2b+2abb+b^2b\]

Using the left distribution of the cube (section \ref{18}), we have $b^2b=b^3$. Thus,
\[a^3+2aba+b^2a+a^2b+2abb+b^2b=a^3+2aba+b^2a+a^2b+2abb+b^3\]

By the commutativity of the addition (section \ref{20}), we get that $b^2a+a^2b=a^2b+b^2a$. This means that:
\[a^3+2aba+b^2a+a^2b+2abb+b^3=a^3+2aba+a^2b+b^2a+2abb+b^3\]

Using the commutativity of the product (section \ref{10}), we have $ba=ab$. This allows us to infer that:
\[a^3+2aba+a^2b+b^2a+2abb+b^3=a^3+2aab+a^2b+b^2a+2abb+b^3\]

From the distribution of the square (section \ref{5}), we have $aa=a^2$. This means that:
\[a^3+2aab+a^2b+b^2a+2abb+b^3=a^3+2a^2b+a^2b+b^2a+2abb+b^3\]

By the left multiplication by the identity (section \ref{12}), we know that $a^2b=1a^2b$. This means that:
\[a^3+2a^2b+a^2b+b^2a+2abb+b^3=a^3+2a^2b+1a^2b+b^2a+2abb+b^3\]

According to the litteral addition (section \ref{13}), we know that $2a^2b+1a^2b=3a^2b$. Thus,
\[a^3+2a^2b+1a^2b+b^2a+2abb+b^3=a^3+3a^2b+b^2a+2abb+b^3\]

By the commutativity of the product (section \ref{10}), we know that $b^2a=ab^2$. This allows us to infer that:
\[a^3+3a^2b+b^2a+2abb+b^3=a^3+3a^2b+ab^2+2abb+b^3\]

By the distribution of the square (section \ref{5}), we know that $bb=b^2$. This allows us to infer that:
\[a^3+3a^2b+ab^2+2abb+b^3=a^3+3a^2b+ab^2+2ab^2+b^3\]

According to the left multiplication by the identity (section \ref{12}), we have $ab^2=1ab^2$. This means that:
\[a^3+3a^2b+ab^2+2ab^2+b^3=a^3+3a^2b+1ab^2+2ab^2+b^3\]

From the litteral addition (section \ref{13}), we get $1ab^2+2ab^2=3ab^2$. Thus,
\[a^3+3a^2b+1ab^2+2ab^2+b^3=a^3+3a^2b+3ab^2+b^3\]

Which therefore allows us to conclude:
\[\left(a+b\right)^3 = a^3+3a^2b+3ab^2+b^3\]
\begin{flushright}
QED
\end{flushright}


\end{document}
